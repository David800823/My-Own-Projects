% Options for packages loaded elsewhere
\PassOptionsToPackage{unicode}{hyperref}
\PassOptionsToPackage{hyphens}{url}
%
\documentclass[
]{article}
\usepackage{amsmath,amssymb}
\usepackage{iftex}
\ifPDFTeX
  \usepackage[T1]{fontenc}
  \usepackage[utf8]{inputenc}
  \usepackage{textcomp} % provide euro and other symbols
\else % if luatex or xetex
  \usepackage{unicode-math} % this also loads fontspec
  \defaultfontfeatures{Scale=MatchLowercase}
  \defaultfontfeatures[\rmfamily]{Ligatures=TeX,Scale=1}
\fi
\usepackage{lmodern}
\ifPDFTeX\else
  % xetex/luatex font selection
\fi
% Use upquote if available, for straight quotes in verbatim environments
\IfFileExists{upquote.sty}{\usepackage{upquote}}{}
\IfFileExists{microtype.sty}{% use microtype if available
  \usepackage[]{microtype}
  \UseMicrotypeSet[protrusion]{basicmath} % disable protrusion for tt fonts
}{}
\makeatletter
\@ifundefined{KOMAClassName}{% if non-KOMA class
  \IfFileExists{parskip.sty}{%
    \usepackage{parskip}
  }{% else
    \setlength{\parindent}{0pt}
    \setlength{\parskip}{6pt plus 2pt minus 1pt}}
}{% if KOMA class
  \KOMAoptions{parskip=half}}
\makeatother
\usepackage{xcolor}
\usepackage[margin=1in]{geometry}
\usepackage{color}
\usepackage{fancyvrb}
\newcommand{\VerbBar}{|}
\newcommand{\VERB}{\Verb[commandchars=\\\{\}]}
\DefineVerbatimEnvironment{Highlighting}{Verbatim}{commandchars=\\\{\}}
% Add ',fontsize=\small' for more characters per line
\usepackage{framed}
\definecolor{shadecolor}{RGB}{248,248,248}
\newenvironment{Shaded}{\begin{snugshade}}{\end{snugshade}}
\newcommand{\AlertTok}[1]{\textcolor[rgb]{0.94,0.16,0.16}{#1}}
\newcommand{\AnnotationTok}[1]{\textcolor[rgb]{0.56,0.35,0.01}{\textbf{\textit{#1}}}}
\newcommand{\AttributeTok}[1]{\textcolor[rgb]{0.13,0.29,0.53}{#1}}
\newcommand{\BaseNTok}[1]{\textcolor[rgb]{0.00,0.00,0.81}{#1}}
\newcommand{\BuiltInTok}[1]{#1}
\newcommand{\CharTok}[1]{\textcolor[rgb]{0.31,0.60,0.02}{#1}}
\newcommand{\CommentTok}[1]{\textcolor[rgb]{0.56,0.35,0.01}{\textit{#1}}}
\newcommand{\CommentVarTok}[1]{\textcolor[rgb]{0.56,0.35,0.01}{\textbf{\textit{#1}}}}
\newcommand{\ConstantTok}[1]{\textcolor[rgb]{0.56,0.35,0.01}{#1}}
\newcommand{\ControlFlowTok}[1]{\textcolor[rgb]{0.13,0.29,0.53}{\textbf{#1}}}
\newcommand{\DataTypeTok}[1]{\textcolor[rgb]{0.13,0.29,0.53}{#1}}
\newcommand{\DecValTok}[1]{\textcolor[rgb]{0.00,0.00,0.81}{#1}}
\newcommand{\DocumentationTok}[1]{\textcolor[rgb]{0.56,0.35,0.01}{\textbf{\textit{#1}}}}
\newcommand{\ErrorTok}[1]{\textcolor[rgb]{0.64,0.00,0.00}{\textbf{#1}}}
\newcommand{\ExtensionTok}[1]{#1}
\newcommand{\FloatTok}[1]{\textcolor[rgb]{0.00,0.00,0.81}{#1}}
\newcommand{\FunctionTok}[1]{\textcolor[rgb]{0.13,0.29,0.53}{\textbf{#1}}}
\newcommand{\ImportTok}[1]{#1}
\newcommand{\InformationTok}[1]{\textcolor[rgb]{0.56,0.35,0.01}{\textbf{\textit{#1}}}}
\newcommand{\KeywordTok}[1]{\textcolor[rgb]{0.13,0.29,0.53}{\textbf{#1}}}
\newcommand{\NormalTok}[1]{#1}
\newcommand{\OperatorTok}[1]{\textcolor[rgb]{0.81,0.36,0.00}{\textbf{#1}}}
\newcommand{\OtherTok}[1]{\textcolor[rgb]{0.56,0.35,0.01}{#1}}
\newcommand{\PreprocessorTok}[1]{\textcolor[rgb]{0.56,0.35,0.01}{\textit{#1}}}
\newcommand{\RegionMarkerTok}[1]{#1}
\newcommand{\SpecialCharTok}[1]{\textcolor[rgb]{0.81,0.36,0.00}{\textbf{#1}}}
\newcommand{\SpecialStringTok}[1]{\textcolor[rgb]{0.31,0.60,0.02}{#1}}
\newcommand{\StringTok}[1]{\textcolor[rgb]{0.31,0.60,0.02}{#1}}
\newcommand{\VariableTok}[1]{\textcolor[rgb]{0.00,0.00,0.00}{#1}}
\newcommand{\VerbatimStringTok}[1]{\textcolor[rgb]{0.31,0.60,0.02}{#1}}
\newcommand{\WarningTok}[1]{\textcolor[rgb]{0.56,0.35,0.01}{\textbf{\textit{#1}}}}
\usepackage{longtable,booktabs,array}
\usepackage{calc} % for calculating minipage widths
% Correct order of tables after \paragraph or \subparagraph
\usepackage{etoolbox}
\makeatletter
\patchcmd\longtable{\par}{\if@noskipsec\mbox{}\fi\par}{}{}
\makeatother
% Allow footnotes in longtable head/foot
\IfFileExists{footnotehyper.sty}{\usepackage{footnotehyper}}{\usepackage{footnote}}
\makesavenoteenv{longtable}
\usepackage{graphicx}
\makeatletter
\def\maxwidth{\ifdim\Gin@nat@width>\linewidth\linewidth\else\Gin@nat@width\fi}
\def\maxheight{\ifdim\Gin@nat@height>\textheight\textheight\else\Gin@nat@height\fi}
\makeatother
% Scale images if necessary, so that they will not overflow the page
% margins by default, and it is still possible to overwrite the defaults
% using explicit options in \includegraphics[width, height, ...]{}
\setkeys{Gin}{width=\maxwidth,height=\maxheight,keepaspectratio}
% Set default figure placement to htbp
\makeatletter
\def\fps@figure{htbp}
\makeatother
\setlength{\emergencystretch}{3em} % prevent overfull lines
\providecommand{\tightlist}{%
  \setlength{\itemsep}{0pt}\setlength{\parskip}{0pt}}
\setcounter{secnumdepth}{-\maxdimen} % remove section numbering
\ifLuaTeX
  \usepackage{selnolig}  % disable illegal ligatures
\fi
\IfFileExists{bookmark.sty}{\usepackage{bookmark}}{\usepackage{hyperref}}
\IfFileExists{xurl.sty}{\usepackage{xurl}}{} % add URL line breaks if available
\urlstyle{same}
\hypersetup{
  pdftitle={Titanic Survival Rate},
  pdfauthor={David Sierra Perez},
  hidelinks,
  pdfcreator={LaTeX via pandoc}}

\title{Titanic Survival Rate}
\author{David Sierra Perez}
\date{2023-07-15}

\begin{document}
\maketitle

\hypertarget{problem}{%
\subsubsection{\texorpdfstring{\textbf{Problem}}{Problem}}\label{problem}}

\begin{itemize}
\tightlist
\item
  What were the highest chances of survival rate on the Titanic based on
  Passenger Class, Age and Sex?
\item
  Is age also the factor affecting the titanic survival rate?
\item
  What is the possible survival rate of a 20 year old man?
\end{itemize}

\hypertarget{what-is-being-tried-to-accomplish-from-this-project}{%
\subsubsection{\texorpdfstring{\textbf{What is being tried to accomplish
from this
project?}}{What is being tried to accomplish from this project?}}\label{what-is-being-tried-to-accomplish-from-this-project}}

The purpose of this project is to learn the basics of machine learning.
We will look at data from a case competition in Kaggle. The competition
is using data from the Titanic to predict survival rates during the
titanic. We will use different data mining techniques to find patterns
in the data.

\hypertarget{introduction}{%
\subsubsection{\texorpdfstring{\textbf{Introduction}}{Introduction}}\label{introduction}}

One of the biggest shipwrecks ever recorded was the Titanic. On April
15, 1912, the Titanic that has been said to be ``Unsinkable'' sank after
colliding with an iceberg. To make things worse, the ship was not
equipped with enough lifeboats for the passengers and crew members. As a
result, over 1500 people died during the incident. Since the Titanic
happened in the 1900s. One can assume that women and kids were given
priority to get on the lift boats. In this project we will dig deeper
into this and try to find out if any other factors played a role in the
people who survived. For instance, if passenger class was a factor and
age as well.

\hypertarget{dataset}{%
\subsubsection{\texorpdfstring{\textbf{Dataset}}{Dataset}}\label{dataset}}

The dataset being used is from Kaggle.com. It contains information from
passenger id, if they survived, ticket class, sex, age, number of
siblings, number of parents, ticket number, passenger fare, cabin number
and lastly port of embark. Kaggle came with 2 data sets. The train data
set that tells us whether the passenger survived or died. Survival is
given with the value of 1. Did not survive was given with the value of
0. In this project, we will only look at Survived, Pclass (passenger
class), Sex, and Age for the the data mining.

\hypertarget{literature-survey-existing-work-on-this-dataset}{%
\subsubsection{\texorpdfstring{\textbf{Literature Survey (existing work
on this
dataset)}}{Literature Survey (existing work on this dataset)}}\label{literature-survey-existing-work-on-this-dataset}}

The Titanic competition has existed for a couple of years now. There
have been more than thousands of projects relating to the ship. A lot of
work has been done on this dataset. Also, there are currently over
14,000 teams competing in the Kaggle competition. The data we retrieved
from Kaggle is more specific and simple. We found out most of the work
being done on this dataset has a detailed statistical analysis along
with machine learning model implementation. But, the data were more
focused on the passenger class and the gender or has related to
different predictors like family, cabin number, age and more. However,
we limited the predictors to passenger class, gender, and age as we
believe that it was an important factor for our project in knowing what
age range possibly survived the most. We gathered the data which was
pre-processed and specifically analyzed the data based on the passenger
class, age and gender.

\hypertarget{analysis-results}{%
\subsubsection{\texorpdfstring{\textbf{Analysis \&
Results}}{Analysis \& Results}}\label{analysis-results}}

Our first step is to load libraries and upload the data into R.

\begin{Shaded}
\begin{Highlighting}[]
\CommentTok{\#install.packages("tidyverse")}
\CommentTok{\#install.packages("readr")}
\CommentTok{\#install.packages("ggplot2")}
\CommentTok{\#install.packages("dplyr")}
\CommentTok{\#install.packages("janitor")}
\CommentTok{\#install.packages("skimr")}
\CommentTok{\#install.packages("here")}

\FunctionTok{library}\NormalTok{(tidyverse)}
\end{Highlighting}
\end{Shaded}

\begin{verbatim}
## -- Attaching core tidyverse packages ------------------------ tidyverse 2.0.0 --
## v dplyr     1.1.2     v readr     2.1.4
## v forcats   1.0.0     v stringr   1.5.0
## v ggplot2   3.4.2     v tibble    3.2.1
## v lubridate 1.9.2     v tidyr     1.3.0
## v purrr     1.0.1     
## -- Conflicts ------------------------------------------ tidyverse_conflicts() --
## x dplyr::filter() masks stats::filter()
## x dplyr::lag()    masks stats::lag()
## i Use the conflicted package (<http://conflicted.r-lib.org/>) to force all conflicts to become errors
\end{verbatim}

\begin{Shaded}
\begin{Highlighting}[]
\FunctionTok{library}\NormalTok{(readr)}
\FunctionTok{library}\NormalTok{(ggplot2)}
\FunctionTok{library}\NormalTok{(dplyr)}
\FunctionTok{library}\NormalTok{(janitor)}
\end{Highlighting}
\end{Shaded}

\begin{verbatim}
## 
## Attaching package: 'janitor'
## 
## The following objects are masked from 'package:stats':
## 
##     chisq.test, fisher.test
\end{verbatim}

\begin{Shaded}
\begin{Highlighting}[]
\FunctionTok{library}\NormalTok{(skimr)}
\FunctionTok{library}\NormalTok{(here)}
\end{Highlighting}
\end{Shaded}

\begin{verbatim}
## here() starts at C:/Users/Sierr/Desktop/My Own Projects/My-Own-Projects
\end{verbatim}

\begin{Shaded}
\begin{Highlighting}[]
\NormalTok{train }\OtherTok{\textless{}{-}} \FunctionTok{read\_csv}\NormalTok{(}\StringTok{"C:/Users/Sierr/Desktop/My Own Projects/My{-}Own{-}Projects/Project/Titanic/Dataset/train.csv"}\NormalTok{)}
\end{Highlighting}
\end{Shaded}

\begin{verbatim}
## Rows: 891 Columns: 12
## -- Column specification --------------------------------------------------------
## Delimiter: ","
## chr (5): Name, Sex, Ticket, Cabin, Embarked
## dbl (7): PassengerId, Survived, Pclass, Age, SibSp, Parch, Fare
## 
## i Use `spec()` to retrieve the full column specification for this data.
## i Specify the column types or set `show_col_types = FALSE` to quiet this message.
\end{verbatim}

Lets look at the first 6 rows of data for train dataset

\begin{Shaded}
\begin{Highlighting}[]
\FunctionTok{head}\NormalTok{(train)}
\end{Highlighting}
\end{Shaded}

\begin{verbatim}
## # A tibble: 6 x 12
##   PassengerId Survived Pclass Name    Sex     Age SibSp Parch Ticket  Fare Cabin
##         <dbl>    <dbl>  <dbl> <chr>   <chr> <dbl> <dbl> <dbl> <chr>  <dbl> <chr>
## 1           1        0      3 Braund~ male     22     1     0 A/5 2~  7.25 <NA> 
## 2           2        1      1 Cuming~ fema~    38     1     0 PC 17~ 71.3  C85  
## 3           3        1      3 Heikki~ fema~    26     0     0 STON/~  7.92 <NA> 
## 4           4        1      1 Futrel~ fema~    35     1     0 113803 53.1  C123 
## 5           5        0      3 Allen,~ male     35     0     0 373450  8.05 <NA> 
## 6           6        0      3 Moran,~ male     NA     0     0 330877  8.46 <NA> 
## # i 1 more variable: Embarked <chr>
\end{verbatim}

Next we will looks at a summary of the data provided

\begin{Shaded}
\begin{Highlighting}[]
\FunctionTok{skim\_without\_charts}\NormalTok{(train)}
\end{Highlighting}
\end{Shaded}

\begin{longtable}[]{@{}ll@{}}
\caption{Data summary}\tabularnewline
\toprule\noalign{}
\endfirsthead
\endhead
\bottomrule\noalign{}
\endlastfoot
Name & train \\
Number of rows & 891 \\
Number of columns & 12 \\
\_\_\_\_\_\_\_\_\_\_\_\_\_\_\_\_\_\_\_\_\_\_\_ & \\
Column type frequency: & \\
character & 5 \\
numeric & 7 \\
\_\_\_\_\_\_\_\_\_\_\_\_\_\_\_\_\_\_\_\_\_\_\_\_ & \\
Group variables & None \\
\end{longtable}

\textbf{Variable type: character}

\begin{longtable}[]{@{}
  >{\raggedright\arraybackslash}p{(\columnwidth - 14\tabcolsep) * \real{0.1944}}
  >{\raggedleft\arraybackslash}p{(\columnwidth - 14\tabcolsep) * \real{0.1389}}
  >{\raggedleft\arraybackslash}p{(\columnwidth - 14\tabcolsep) * \real{0.1944}}
  >{\raggedleft\arraybackslash}p{(\columnwidth - 14\tabcolsep) * \real{0.0556}}
  >{\raggedleft\arraybackslash}p{(\columnwidth - 14\tabcolsep) * \real{0.0556}}
  >{\raggedleft\arraybackslash}p{(\columnwidth - 14\tabcolsep) * \real{0.0833}}
  >{\raggedleft\arraybackslash}p{(\columnwidth - 14\tabcolsep) * \real{0.1250}}
  >{\raggedleft\arraybackslash}p{(\columnwidth - 14\tabcolsep) * \real{0.1528}}@{}}
\toprule\noalign{}
\begin{minipage}[b]{\linewidth}\raggedright
skim\_variable
\end{minipage} & \begin{minipage}[b]{\linewidth}\raggedleft
n\_missing
\end{minipage} & \begin{minipage}[b]{\linewidth}\raggedleft
complete\_rate
\end{minipage} & \begin{minipage}[b]{\linewidth}\raggedleft
min
\end{minipage} & \begin{minipage}[b]{\linewidth}\raggedleft
max
\end{minipage} & \begin{minipage}[b]{\linewidth}\raggedleft
empty
\end{minipage} & \begin{minipage}[b]{\linewidth}\raggedleft
n\_unique
\end{minipage} & \begin{minipage}[b]{\linewidth}\raggedleft
whitespace
\end{minipage} \\
\midrule\noalign{}
\endhead
\bottomrule\noalign{}
\endlastfoot
Name & 0 & 1.00 & 12 & 82 & 0 & 891 & 0 \\
Sex & 0 & 1.00 & 4 & 6 & 0 & 2 & 0 \\
Ticket & 0 & 1.00 & 3 & 18 & 0 & 681 & 0 \\
Cabin & 687 & 0.23 & 1 & 15 & 0 & 147 & 0 \\
Embarked & 2 & 1.00 & 1 & 1 & 0 & 3 & 0 \\
\end{longtable}

\textbf{Variable type: numeric}

\begin{longtable}[]{@{}
  >{\raggedright\arraybackslash}p{(\columnwidth - 18\tabcolsep) * \real{0.1667}}
  >{\raggedleft\arraybackslash}p{(\columnwidth - 18\tabcolsep) * \real{0.1190}}
  >{\raggedleft\arraybackslash}p{(\columnwidth - 18\tabcolsep) * \real{0.1667}}
  >{\raggedleft\arraybackslash}p{(\columnwidth - 18\tabcolsep) * \real{0.0833}}
  >{\raggedleft\arraybackslash}p{(\columnwidth - 18\tabcolsep) * \real{0.0833}}
  >{\raggedleft\arraybackslash}p{(\columnwidth - 18\tabcolsep) * \real{0.0595}}
  >{\raggedleft\arraybackslash}p{(\columnwidth - 18\tabcolsep) * \real{0.0833}}
  >{\raggedleft\arraybackslash}p{(\columnwidth - 18\tabcolsep) * \real{0.0833}}
  >{\raggedleft\arraybackslash}p{(\columnwidth - 18\tabcolsep) * \real{0.0714}}
  >{\raggedleft\arraybackslash}p{(\columnwidth - 18\tabcolsep) * \real{0.0833}}@{}}
\toprule\noalign{}
\begin{minipage}[b]{\linewidth}\raggedright
skim\_variable
\end{minipage} & \begin{minipage}[b]{\linewidth}\raggedleft
n\_missing
\end{minipage} & \begin{minipage}[b]{\linewidth}\raggedleft
complete\_rate
\end{minipage} & \begin{minipage}[b]{\linewidth}\raggedleft
mean
\end{minipage} & \begin{minipage}[b]{\linewidth}\raggedleft
sd
\end{minipage} & \begin{minipage}[b]{\linewidth}\raggedleft
p0
\end{minipage} & \begin{minipage}[b]{\linewidth}\raggedleft
p25
\end{minipage} & \begin{minipage}[b]{\linewidth}\raggedleft
p50
\end{minipage} & \begin{minipage}[b]{\linewidth}\raggedleft
p75
\end{minipage} & \begin{minipage}[b]{\linewidth}\raggedleft
p100
\end{minipage} \\
\midrule\noalign{}
\endhead
\bottomrule\noalign{}
\endlastfoot
PassengerId & 0 & 1.0 & 446.00 & 257.35 & 1.00 & 223.50 & 446.00 & 668.5
& 891.00 \\
Survived & 0 & 1.0 & 0.38 & 0.49 & 0.00 & 0.00 & 0.00 & 1.0 & 1.00 \\
Pclass & 0 & 1.0 & 2.31 & 0.84 & 1.00 & 2.00 & 3.00 & 3.0 & 3.00 \\
Age & 177 & 0.8 & 29.70 & 14.53 & 0.42 & 20.12 & 28.00 & 38.0 & 80.00 \\
SibSp & 0 & 1.0 & 0.52 & 1.10 & 0.00 & 0.00 & 0.00 & 1.0 & 8.00 \\
Parch & 0 & 1.0 & 0.38 & 0.81 & 0.00 & 0.00 & 0.00 & 0.0 & 6.00 \\
Fare & 0 & 1.0 & 32.20 & 49.69 & 0.00 & 7.91 & 14.45 & 31.0 & 512.33 \\
\end{longtable}

From the Summary found using ``skim\_without\_charts()'' we noticed a
couple of things * There is 891 observations (rows) * There are 12
columns * 5 columns are string/character variables * 7 columns are
integers * we have missing values for Cabin, Embarked and Age.

\hypertarget{cleaning-data}{%
\subsubsection{\texorpdfstring{\textbf{Cleaning
Data}}{Cleaning Data}}\label{cleaning-data}}

Our analysis is based on the following columns: * PasssengerId *
Survived * Pclass * Age * Sex

The other columns we will drop.

\begin{Shaded}
\begin{Highlighting}[]
\NormalTok{keep\_columns }\OtherTok{=} \FunctionTok{c}\NormalTok{(}\StringTok{"PassengerId"}\NormalTok{, }\StringTok{"Survived"}\NormalTok{, }\StringTok{"Pclass"}\NormalTok{, }\StringTok{"Age"}\NormalTok{, }\StringTok{"Sex"}\NormalTok{)}

\NormalTok{cleaned\_train }\OtherTok{\textless{}{-}}\NormalTok{ train[keep\_columns]}

\FunctionTok{skim\_without\_charts}\NormalTok{(cleaned\_train)}
\end{Highlighting}
\end{Shaded}

\begin{longtable}[]{@{}ll@{}}
\caption{Data summary}\tabularnewline
\toprule\noalign{}
\endfirsthead
\endhead
\bottomrule\noalign{}
\endlastfoot
Name & cleaned\_train \\
Number of rows & 891 \\
Number of columns & 5 \\
\_\_\_\_\_\_\_\_\_\_\_\_\_\_\_\_\_\_\_\_\_\_\_ & \\
Column type frequency: & \\
character & 1 \\
numeric & 4 \\
\_\_\_\_\_\_\_\_\_\_\_\_\_\_\_\_\_\_\_\_\_\_\_\_ & \\
Group variables & None \\
\end{longtable}

\textbf{Variable type: character}

\begin{longtable}[]{@{}
  >{\raggedright\arraybackslash}p{(\columnwidth - 14\tabcolsep) * \real{0.1944}}
  >{\raggedleft\arraybackslash}p{(\columnwidth - 14\tabcolsep) * \real{0.1389}}
  >{\raggedleft\arraybackslash}p{(\columnwidth - 14\tabcolsep) * \real{0.1944}}
  >{\raggedleft\arraybackslash}p{(\columnwidth - 14\tabcolsep) * \real{0.0556}}
  >{\raggedleft\arraybackslash}p{(\columnwidth - 14\tabcolsep) * \real{0.0556}}
  >{\raggedleft\arraybackslash}p{(\columnwidth - 14\tabcolsep) * \real{0.0833}}
  >{\raggedleft\arraybackslash}p{(\columnwidth - 14\tabcolsep) * \real{0.1250}}
  >{\raggedleft\arraybackslash}p{(\columnwidth - 14\tabcolsep) * \real{0.1528}}@{}}
\toprule\noalign{}
\begin{minipage}[b]{\linewidth}\raggedright
skim\_variable
\end{minipage} & \begin{minipage}[b]{\linewidth}\raggedleft
n\_missing
\end{minipage} & \begin{minipage}[b]{\linewidth}\raggedleft
complete\_rate
\end{minipage} & \begin{minipage}[b]{\linewidth}\raggedleft
min
\end{minipage} & \begin{minipage}[b]{\linewidth}\raggedleft
max
\end{minipage} & \begin{minipage}[b]{\linewidth}\raggedleft
empty
\end{minipage} & \begin{minipage}[b]{\linewidth}\raggedleft
n\_unique
\end{minipage} & \begin{minipage}[b]{\linewidth}\raggedleft
whitespace
\end{minipage} \\
\midrule\noalign{}
\endhead
\bottomrule\noalign{}
\endlastfoot
Sex & 0 & 1 & 4 & 6 & 0 & 2 & 0 \\
\end{longtable}

\textbf{Variable type: numeric}

\begin{longtable}[]{@{}
  >{\raggedright\arraybackslash}p{(\columnwidth - 18\tabcolsep) * \real{0.1772}}
  >{\raggedleft\arraybackslash}p{(\columnwidth - 18\tabcolsep) * \real{0.1266}}
  >{\raggedleft\arraybackslash}p{(\columnwidth - 18\tabcolsep) * \real{0.1772}}
  >{\raggedleft\arraybackslash}p{(\columnwidth - 18\tabcolsep) * \real{0.0886}}
  >{\raggedleft\arraybackslash}p{(\columnwidth - 18\tabcolsep) * \real{0.0886}}
  >{\raggedleft\arraybackslash}p{(\columnwidth - 18\tabcolsep) * \real{0.0633}}
  >{\raggedleft\arraybackslash}p{(\columnwidth - 18\tabcolsep) * \real{0.0886}}
  >{\raggedleft\arraybackslash}p{(\columnwidth - 18\tabcolsep) * \real{0.0506}}
  >{\raggedleft\arraybackslash}p{(\columnwidth - 18\tabcolsep) * \real{0.0759}}
  >{\raggedleft\arraybackslash}p{(\columnwidth - 18\tabcolsep) * \real{0.0633}}@{}}
\toprule\noalign{}
\begin{minipage}[b]{\linewidth}\raggedright
skim\_variable
\end{minipage} & \begin{minipage}[b]{\linewidth}\raggedleft
n\_missing
\end{minipage} & \begin{minipage}[b]{\linewidth}\raggedleft
complete\_rate
\end{minipage} & \begin{minipage}[b]{\linewidth}\raggedleft
mean
\end{minipage} & \begin{minipage}[b]{\linewidth}\raggedleft
sd
\end{minipage} & \begin{minipage}[b]{\linewidth}\raggedleft
p0
\end{minipage} & \begin{minipage}[b]{\linewidth}\raggedleft
p25
\end{minipage} & \begin{minipage}[b]{\linewidth}\raggedleft
p50
\end{minipage} & \begin{minipage}[b]{\linewidth}\raggedleft
p75
\end{minipage} & \begin{minipage}[b]{\linewidth}\raggedleft
p100
\end{minipage} \\
\midrule\noalign{}
\endhead
\bottomrule\noalign{}
\endlastfoot
PassengerId & 0 & 1.0 & 446.00 & 257.35 & 1.00 & 223.50 & 446 & 668.5 &
891 \\
Survived & 0 & 1.0 & 0.38 & 0.49 & 0.00 & 0.00 & 0 & 1.0 & 1 \\
Pclass & 0 & 1.0 & 2.31 & 0.84 & 1.00 & 2.00 & 3 & 3.0 & 3 \\
Age & 177 & 0.8 & 29.70 & 14.53 & 0.42 & 20.12 & 28 & 38.0 & 80 \\
\end{longtable}

\begin{Shaded}
\begin{Highlighting}[]
\NormalTok{cleaned\_train }\OtherTok{\textless{}{-}}\NormalTok{ cleaned\_train }\SpecialCharTok{\%\textgreater{}\%} 
  \FunctionTok{drop\_na}\NormalTok{()}
\end{Highlighting}
\end{Shaded}

Before we can do further analysis, we need to convert Sex into a boolean
variable. To do this i am choosing to name the following genders with
the following number. * Male = 1 * Female = 0

\begin{Shaded}
\begin{Highlighting}[]
\NormalTok{cleaned\_train}\SpecialCharTok{$}\NormalTok{Sex[cleaned\_train}\SpecialCharTok{$}\NormalTok{Sex }\SpecialCharTok{==} \StringTok{"male"}\NormalTok{] }\OtherTok{\textless{}{-}} \DecValTok{1}      
\NormalTok{cleaned\_train}\SpecialCharTok{$}\NormalTok{Sex[cleaned\_train}\SpecialCharTok{$}\NormalTok{Sex }\SpecialCharTok{==} \StringTok{"female"}\NormalTok{] }\OtherTok{\textless{}{-}} \DecValTok{0}  

\FunctionTok{head}\NormalTok{(cleaned\_train)}
\end{Highlighting}
\end{Shaded}

\begin{verbatim}
## # A tibble: 6 x 5
##   PassengerId Survived Pclass   Age Sex  
##         <dbl>    <dbl>  <dbl> <dbl> <chr>
## 1           1        0      3    22 1    
## 2           2        1      1    38 0    
## 3           3        1      3    26 0    
## 4           4        1      1    35 0    
## 5           5        0      3    35 1    
## 6           7        0      1    54 1
\end{verbatim}

\hypertarget{now-we-will-do-a-quick-analysis-of-our-data}{%
\subsubsection{\texorpdfstring{\textbf{Now we will do a quick analysis
of our
data}}{Now we will do a quick analysis of our data}}\label{now-we-will-do-a-quick-analysis-of-our-data}}

The first step of our analysis was to get a general count of Sex,
Passenger Class, Age, and whether someone survived from the train data.

We start of with SEX.

\begin{Shaded}
\begin{Highlighting}[]
\FunctionTok{ggplot}\NormalTok{(cleaned\_train, }\FunctionTok{aes}\NormalTok{(}\AttributeTok{x =}\NormalTok{ Sex))}\SpecialCharTok{+} 
  \FunctionTok{geom\_bar}\NormalTok{(}\AttributeTok{fill =} \StringTok{"steelblue"}\NormalTok{) }\SpecialCharTok{+} 
  \FunctionTok{labs}\NormalTok{(}\StringTok{"text"}\NormalTok{, }\AttributeTok{title =} \StringTok{"Total Count of Each Gender"}\NormalTok{) }\SpecialCharTok{+}
  \FunctionTok{geom\_text}\NormalTok{(}\FunctionTok{aes}\NormalTok{(}\AttributeTok{label =}\NormalTok{ ..count..), }\AttributeTok{stat =} \StringTok{"count"}\NormalTok{, }\AttributeTok{vjust =} \FloatTok{1.5}\NormalTok{, }\AttributeTok{color =} \StringTok{"white"}\NormalTok{) }\SpecialCharTok{+}
  \FunctionTok{theme\_classic}\NormalTok{()}
\end{Highlighting}
\end{Shaded}

\begin{verbatim}
## Warning: The dot-dot notation (`..count..`) was deprecated in ggplot2 3.4.0.
## i Please use `after_stat(count)` instead.
## This warning is displayed once every 8 hours.
## Call `lifecycle::last_lifecycle_warnings()` to see where this warning was
## generated.
\end{verbatim}

\includegraphics{Titanic_Project_R_Markdown_files/figure-latex/Barplot Gender-1.pdf}

Here is a graph for Passenger Class.

\begin{Shaded}
\begin{Highlighting}[]
\FunctionTok{ggplot}\NormalTok{(cleaned\_train, }\FunctionTok{aes}\NormalTok{(}\AttributeTok{x =}\NormalTok{ Pclass))}\SpecialCharTok{+} 
  \FunctionTok{geom\_bar}\NormalTok{(}\AttributeTok{fill =} \StringTok{"steelblue"}\NormalTok{) }\SpecialCharTok{+} 
  \FunctionTok{labs}\NormalTok{(}\StringTok{"text"}\NormalTok{, }\AttributeTok{title =} \StringTok{"Total Count for Each Passenger Class"}\NormalTok{) }\SpecialCharTok{+}
  \FunctionTok{geom\_text}\NormalTok{(}\FunctionTok{aes}\NormalTok{(}\AttributeTok{label =}\NormalTok{ ..count..), }\AttributeTok{stat =} \StringTok{"count"}\NormalTok{, }\AttributeTok{vjust =} \FloatTok{1.5}\NormalTok{, }\AttributeTok{color =} \StringTok{"white"}\NormalTok{) }\SpecialCharTok{+}
  \FunctionTok{theme\_classic}\NormalTok{()}
\end{Highlighting}
\end{Shaded}

\includegraphics{Titanic_Project_R_Markdown_files/figure-latex/Barplot Pclass-1.pdf}

\begin{Shaded}
\begin{Highlighting}[]
\FunctionTok{ggplot}\NormalTok{(cleaned\_train, }\FunctionTok{aes}\NormalTok{(}\AttributeTok{x=}\NormalTok{Age)) }\SpecialCharTok{+} 
  \FunctionTok{geom\_histogram}\NormalTok{(}\AttributeTok{binwidth=}\DecValTok{1}\NormalTok{, }\AttributeTok{fill =} \StringTok{"steelblue"}\NormalTok{ ) }\SpecialCharTok{+} 
  \FunctionTok{labs}\NormalTok{(}\StringTok{"text"}\NormalTok{, }\AttributeTok{title =} \StringTok{"Distribution of Age"}\NormalTok{) }\SpecialCharTok{+}
  \FunctionTok{theme\_classic}\NormalTok{()}
\end{Highlighting}
\end{Shaded}

\includegraphics{Titanic_Project_R_Markdown_files/figure-latex/Barplot Age-1.pdf}

\begin{Shaded}
\begin{Highlighting}[]
\FunctionTok{ggplot}\NormalTok{(cleaned\_train, }\FunctionTok{aes}\NormalTok{(}\AttributeTok{x =}\NormalTok{ Survived))}\SpecialCharTok{+} 
  \FunctionTok{geom\_bar}\NormalTok{(}\AttributeTok{fill =} \StringTok{"steelblue"}\NormalTok{) }\SpecialCharTok{+} 
  \FunctionTok{labs}\NormalTok{(}\StringTok{"text"}\NormalTok{, }\AttributeTok{title =} \StringTok{"Total Count of Each Gender"}\NormalTok{, }\AttributeTok{subtitle =} \StringTok{"0 = Did not Survive | 1 = Survived"}\NormalTok{) }\SpecialCharTok{+}
  \FunctionTok{geom\_text}\NormalTok{(}\FunctionTok{aes}\NormalTok{(}\AttributeTok{label =}\NormalTok{ ..count..), }\AttributeTok{stat =} \StringTok{"count"}\NormalTok{, }\AttributeTok{vjust =} \FloatTok{5.5}\NormalTok{, }\AttributeTok{color =} \StringTok{"white"}\NormalTok{) }\SpecialCharTok{+}
  \FunctionTok{theme\_classic}\NormalTok{()}
\end{Highlighting}
\end{Shaded}

\includegraphics{Titanic_Project_R_Markdown_files/figure-latex/Barplot Survived-1.pdf}

A couple points from the first 4 graphs: * There are more males than
females in the dataset * There are more passngers in class 3 * A
majority of people are in the ages between 15 to 45 * More people died
than survived

\hypertarget{next-lets-see-even-more-analysis}{%
\subsection{\texorpdfstring{\textbf{Next lets see even more
Analysis}}{Next lets see even more Analysis}}\label{next-lets-see-even-more-analysis}}

\begin{Shaded}
\begin{Highlighting}[]
\FunctionTok{ggplot}\NormalTok{(cleaned\_train, }\FunctionTok{aes}\NormalTok{(}\AttributeTok{x =}\NormalTok{ Survived, }\AttributeTok{fill =}\NormalTok{ Sex))}\SpecialCharTok{+} 
 \FunctionTok{geom\_bar}\NormalTok{() }\SpecialCharTok{+} 
  \FunctionTok{geom\_text}\NormalTok{(}\FunctionTok{aes}\NormalTok{(}\AttributeTok{label =}\NormalTok{ ..count..), }\AttributeTok{stat =} \StringTok{"count"}\NormalTok{, }\AttributeTok{hjust =} \DecValTok{1}\NormalTok{, }\AttributeTok{vjust =} \FloatTok{1.5}\NormalTok{, }\AttributeTok{color =} \StringTok{"white"}\NormalTok{, }\AttributeTok{position =} \StringTok{"stack"}\NormalTok{)}\SpecialCharTok{+}
  \FunctionTok{labs}\NormalTok{(}\StringTok{"text"}\NormalTok{, }\AttributeTok{title =} \StringTok{"Survivability by Gender"}\NormalTok{, }\AttributeTok{subtitle =} \StringTok{"0 = Did not Survive | 1 = Survived"}\NormalTok{)}
\end{Highlighting}
\end{Shaded}

\includegraphics{Titanic_Project_R_Markdown_files/figure-latex/Barplot S/F-1.pdf}
\#\# \textbf{Decision Tree} We will see try to predict the survival rate
of the passenger by Sex, Age, and Passenger Class. Lets take a look at
our data again.

\begin{Shaded}
\begin{Highlighting}[]
\NormalTok{cleaned\_train}
\end{Highlighting}
\end{Shaded}

\begin{verbatim}
## # A tibble: 714 x 5
##    PassengerId Survived Pclass   Age Sex  
##          <dbl>    <dbl>  <dbl> <dbl> <chr>
##  1           1        0      3    22 1    
##  2           2        1      1    38 0    
##  3           3        1      3    26 0    
##  4           4        1      1    35 0    
##  5           5        0      3    35 1    
##  6           7        0      1    54 1    
##  7           8        0      3     2 1    
##  8           9        1      3    27 0    
##  9          10        1      2    14 0    
## 10          11        1      3     4 0    
## # i 704 more rows
\end{verbatim}

Lets look at a decision tree.

\begin{Shaded}
\begin{Highlighting}[]
\CommentTok{\#install.packages("tree")}
\FunctionTok{library}\NormalTok{(tree)}
\NormalTok{dec\_tree }\OtherTok{\textless{}{-}} \FunctionTok{tree}\NormalTok{(Survived }\SpecialCharTok{\textasciitilde{}}\NormalTok{ Sex }\SpecialCharTok{+}\NormalTok{ Pclass }\SpecialCharTok{+}\NormalTok{ Age, }\AttributeTok{data =}\NormalTok{ cleaned\_train)}

\FunctionTok{summary}\NormalTok{(dec\_tree)}
\end{Highlighting}
\end{Shaded}

\begin{verbatim}
## 
## Regression tree:
## tree(formula = Survived ~ Sex + Pclass + Age, data = cleaned_train)
## Number of terminal nodes:  8 
## Residual mean deviance:  0.1295 = 91.4 / 706 
## Distribution of residuals:
##    Min. 1st Qu.  Median    Mean 3rd Qu.    Max. 
## -0.9434 -0.1182 -0.1182  0.0000  0.0566  0.9167
\end{verbatim}

\begin{Shaded}
\begin{Highlighting}[]
\FunctionTok{plot}\NormalTok{(dec\_tree)}
\FunctionTok{text}\NormalTok{(dec\_tree, }\AttributeTok{pretty =} \DecValTok{1}\NormalTok{)}
\end{Highlighting}
\end{Shaded}

\includegraphics{Titanic_Project_R_Markdown_files/figure-latex/Decision Tree-1.pdf}

\end{document}
